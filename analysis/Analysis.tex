\documentclass{article}
\usepackage{amsmath}
\usepackage{amssymb}
\usepackage{amsfonts}
\usepackage{enumitem}
\usepackage[hidelinks]{hyperref}
\usepackage[margin=1in]{geometry}

\setcounter{tocdepth}{2}

\begin{document}

\title{Analysis}
\date{}
\maketitle

\newpage
\tableofcontents

\newpage
\section{Sets}

\textbf{Definition:} A set is a collection of objects, called the \emph{elements} or \emph{members} of the set.  
We write $x \in X$ if $x$ is an element of the set $X$ and $x \notin X$ if $x$ is not an element of $X$.

Two sets $X = Y$, if
\[
x \in X \iff x \in Y
\]
(``iff'' or ``$\Leftrightarrow$'' both mean ``if and only if'').

The empty set is denoted by $\varnothing$, that is, the set without any elements.  
$X$ is \emph{nonempty} if it has at least one element.

We can define sets by listing their elements:
\[
X = \{a, b, c, d\}.
\]

We can also have infinite sets, for example:
\[
\mathbb{N} = \{1, 2, 3, \dots\}, \quad
\mathbb{N}_0 = \{0, 1, 2, 3, \dots\},
\]
\[
\mathbb{Z} = \{\dots, -3, -2, -1, 0, 1, 2, 3, \dots\}, \quad
\mathbb{Q} = \left\{ \tfrac{p}{q} : p, q \in \mathbb{Z}, \; q \neq 0 \right\},
\]
\[
\mathbb{R} = \{\text{all numbers with decimal expansions}\}.
\]

\subsection{Subsets}

$A$ is a subset of a set $X$ or $A$ is included in $X$, written $A \subseteq X$, if every element of $A$ belongs to $X$. $A$ is a proper subset of $X$, written as $A \subset X$, when $A \subseteq X$, but $A \neq X$.

\textbf{Def.:} The power set $\mathcal{P}$($X$) of a set $X$ is the set of all subsets of $X$.

\textbf{Ex.:} $X=$\{1,2,3\}, then $\mathcal{P}$($X$)=\{$\varnothing$,\{1\},\{2\},\{3\},\{1,2\},\{1,3\},\{2,3\},\{1,2,3\}\}

The power set $\mathcal{P}$($X$) of a set $X$ with |$X$|=$n$ elements has |$\mathcal{P}$($X$)|=$2^n$ elements because, in defining a subset, we have two independent choices for each element (does it belong to the subset or not?). Thus, the notation $2^X=\mathcal{P}(X)$ is also in use.

\subsection{Set operations}

The intersection $A\cap B$ of two sets $A$, $B$ is the set of all elements that belong to both $A$ and $B$. Two sets $A$, $B$ are said to be disjoint if $A\cap B = \varnothing $; that is, if $A$ and $B$ have no elements in common.

The union $A \cup B$ is the set of all elements that belong to $A$ or $B$. Note that we always use ‘or’ in an inclusive sense, so that $x \in A \cup B$ if $x$ is an element of $A$ or $B$, or both $A$ and $B$. (Thus, $A \cap B \subseteq A \cup B$.)

The set-difference of two sets $B$ and $A$ is the set of elements of $B$ that do not belong to $A$, that is $B \setminus A = \{x \in B:x \notin A\}$.
If we consider sets that are subsets of a fixed set $X$ (called the universe) that is understood from the context, then we write $A^c = \overline{A} = X \setminus A$ to denote the complement of $A \subseteq X$ in $X$. Note that $(A^c)^c = A$.

The Cartesian product $A \times B$ of sets $A$, $B$ is the set whose members all possible ordered pairs $(a, b)$ with $a\in A$, $b\in B$, thus $A\times B = \{(a,b):a\in A, b\in B \}$ and $|A\times B| = |A||B|$.

\subsection{Algebraic properties}

Intersection is a commutative operation $A\cap B = B \cap A$; and an \textit{associative} operation, that is:
\[
(A\cap B)\cap C = A \cap (B\cap C)\text{, thus} = A\cap B \cap C
\]
   both are also true for the union $A\cup B$.
Intersection distributes over union and union distributes over intersection:
\[
A\cap (B\cup C) = (A\cap B)\cup (A\cap C)
\]
\[
A\cup (B\cap C) = (A\cup B) \cap (A\cup C)
\]
We have De Morgan’s laws:
\[
(a\cup B)^c = A^c\cap B^c \text{ and } (A\cap B)^c = A^c\cup B^c
\]
Arbitrary many unions and intersections: Let $\mathcal{C}$ be a collection of sets. Then
\[
\bigcup\mathcal{C} = \bigcup_{A\in  \mathcal{C}}A = \{x: x \in A\text{, for some } A \in \mathcal{C}\}
\]
\[
\bigcap\mathcal{C} = \bigcap_{A\in  \mathcal{C}}A = \{x: x \in A\text{, for all } A \in \mathcal{C}\}
\]

\subsection{Relations}

Any subset of the Cartesian product of two sets $X$, $Y$ defines a (binary) relation $R \subseteq X \times Y $ between these two sets. Given $(x, y) \in R$ we may denote this inclusion simply as $xRy$. 
   Notation: $\forall$ means ‘for all’, $\exists$ means ‘exists’.
A binary relation $R$ is \textit{univalent} if
\[
\forall x \in X, \forall y \in Y, \forall z \in Y \text{ we have } ((x, y) \in R \text{ and } (x, z)\in R) \implies y = z
\]

A binary relation $R$ is \textit{total} if
\[
\forall x \in X, \exists y \in Y \text{ we have } (x,y) \in R
\]

\textbf{Def.:} A partially defined function is a univalent binary relation, and a function is a univalent and total binary relation. Thus a function $f: X \mapsto Y$ is defined by a univalent and total $xRy\iff y = f(x)$.
The set of all functions from $X$ to $Y$ is commonly denoted as 
\[
Y^X = \prod_{x\in X}Y
\]

\subsection{Orders and equivalences}

\textbf{Def.:} An order $\leq$ on a set $X$ is a binary relation on $X$, s.t. for every $x, y, z \in X$:
\begin{enumerate}
    \item $x \leq x$ (reflexivity),
    \item If $x \leq y$ and $y \leq x$ then $x=y$ (antisymmetry),
    \item If $x \leq y$ and $y\leq z$ then $x\leq z$ (transitivity).
\end{enumerate}


An order is \textit{linear} or \textit{total} if $\forall x,y\in X$ either $x\leq y$ or $y\leq x$. If $\leq$ is an order, then we define a strict order by $x < y$ if $x\leq y$ and $x\neq y$.

If for a relation $\sim$ in 2. instead of antisymmetry we have \textit{symmetry}: 
If $x\sim y$ then $y\sim x$
then $\sim$ is called an equivalence relation.

\section{Functions}

Per definition a function $f: X\mapsto Y$ is a univalent and total relation, that is for every $x\in X$ there is a unique $y = f(x)\in Y$. $Do(f) = X$ is called the \textit{domain} of $f$, and $Ran(f) = \{y \in Y: \exists x \in X, y = f(x)\}\subseteq Y$ is called the range of $f$. Also $f(A) = \{y\in Y:\exists x\in A, y = f(x)\}$ for some $A\subseteq X$.

\textbf{Ex.:} The identity function $id_X:X$ on a set $X$ is the function that maps every element of $X$ to itself, that is $id_X(x) = x$ for all $x\in X$.

\textbf{Ex.:} the characteristic or indicator function $\chi_A:X\mapsto \{0, 1\}$ of $A\subseteq X$ is defined as 
\[
\chi_{A}(x) =
\begin{cases}
1, & x \in A,\\
0, & x \notin A.
\end{cases}
\]
The graph of a function $f: X\mapsto Y$ is defined as
\[
G_f= \{(x, y)\in X\times Y:y = f(x)\}
\]

\subsection{Properties of functions}

A function $f: X\mapsto Y$ is
\begin{enumerate}
    \item injective (one-to-one) if it maps distinct elements to distinct elements, that is $x_1, x_2\in X$ and $x_1\neq x_2$ implies that $f(x_1)\neq f(x_2)$,
    \item surjective (onto) if its range $Ran(f)$ = Y, that is for every $y$ there exists an $x$, s.t. $y=f(x)$.
    \item If a function is both injective and surjective then its bijective.
\end{enumerate}

We define the composition $f\circ g(z) = f(g(z))$ of functions $f: Y\mapsto X$ and $g: Z\mapsto Y$. Note that we need the inclusion $Ran(g)\subseteq Do(f)$. $\circ$ is associative.

A bijective function $f: X\mapsto Y$ has an inverse $f^{-1}: Y\mapsto X$ defined by
\[
f^{-1}(y) = x \text{ if and only if } f(x) = y
\]
   that is $f\circ f^{-1} = id_Y$ and $f^{-1}\circ f = id_X$.
   
If $f: x\mapsto Y$ is merely injective than still $f: X\mapsto Ran(f)$ is bijective, thus invertible on its range with inverse $f^{-1}: Ran(f)\mapsto X$.

\subsection{Groups, monoids, fields}

\textbf{Def.:} Given a function $f: X\times X \mapsto X$ we may denote $f(x, y) = x * y$ and consider this as a binary operation on $X$. For example addition of integers is such an operation. The we say that * is/has

\begin{enumerate}
    \item Associative, if $x * (y * z) = (x * y) * z$,
    \item Commutative, if $x * y = y * x$,
    \item Neutral element, if there exists $e\in X$ (a neutral element), s.t. $x*e=e*x=x$,
    \item Inverse elements, if for all $x\in X$ there exists $x'\in X$ called an inverse of $x$, s.t. $x*x'=x'*x =e$ where $e$ is a neutral element.
\end{enumerate}

\textbf{Def.:} $(X, *)$ is called a
\begin{enumerate}
    \item Semigroup, if * is associative,
    \item Monoid, if $(X, *)$ is a semigroup and has a neutral element,
    \item Group, if $(X, *)$ is a monoid and every element $x\in X$ has an inverse.
\end{enumerate}

\textbf{Theorem:} In a group $(X, *)$ the neutral element $e\in X$ and inverse $x'$ for any fixed $x\in X$ are unique.

\textbf{Proof.:} Indeed, if there would be two neutral elements $e, e'$, then $e'=e*e'=e$. Also assuming $x*y=e=x*z$, implies $x'*(x*y)=x'*(x*z)$, that is $y=z=x'$.

Let $(X, \cdot, +)$ be given with binary operations $\cdot$ and $+$.

\textbf{Def.:} $(X, \cdot, +)$ is a field if
\begin{enumerate}
    \item $(X, \cdot, +)$ is a commutative group with neutral element 0,
    \item $(X\setminus \{0\}, \cdot)$ is a commutative group,
    \item $\cdot$ distributes over $+$ (distributivity), that is: $x\cdot (y+z) = x\cdot y + x\cdot z$.
In this case $+$ is usually called addition and $\cdot$ multiplication.
\end{enumerate}

\textbf{Ex.:} the rational numbers $\mathbb{Q}$ is a field, moreover an ordered field $(\mathbb{Q}, \cdot, +, \leq)$ equipped with the total order $x\leq y\iff0\leq y-x$, where 0 is the neutral element of $+$.

\textbf{Ex.:} the set of real numbers $\mathbb{R}$ is also a totally ordered field $(\mathbb{R}, \cdot, +, \leq)$

\textbf{Axiom:} the $\leq$ order of $\mathbb{R}$ satisfies
\begin{enumerate}[label=\Roman*.]
  \item $x\leq y$ implies $x+z\leq y+z$,
  \item $x< y$ and $z>0$ implies $xz<yz$.
\end{enumerate}

\subsection{Supremum and infimum}

\textbf{Def.:} A set $A\subseteq \mathbb{R}$ is \textit{bounded} \textit{from above}, if $\exists M\in \mathbb{R}$ s.t.  $x\leq M$ for all $x\in A$; and it is \textit{bounded from below}, if $\exists m\in \mathbb{R}$ s.t.  $x\geq m$ for all $x\in A$. If both holds for $A$, then it is bounded.				(‘s.t.’ is short hand for ‘such that’)

\textbf{Def.:} If $M\in \mathbb{R}$ is an upper bound of $A\subseteq \mathbb{R}$ s.t. for any other upper bound $M'\in \mathbb{R}$ of $A$ we have $M\leq M'$, then $M$ is called the least upper bound of $A$, denoted as
\[
M=\sup A
\]
Similarly, the greatest lower bound of $A\subseteq \mathbb{R}$, if exists, is denoted by 
\[
m=\inf A
\]
  meaning $m\geq m'$ for any lower bound $m$ of $A$.
If $A=\{x_i: i\in I\}\subseteq \mathbb{R}$ for an index set $I$, we also write:
\[
\sup A=\sup_{i\in I}x_i \text{ and } \inf A = \inf_{i\in I}x_i
\]

\textbf{Fact:} by the definition supremum and infimum of a set, if they exist, are both unique and $A\geq \inf A$ for nonempty $A\subseteq \mathbb{R}$.

\textbf{Def.:} if $\sup A\in A$, then we call it the maximum of $A$ denoted by $\max A$, similarly if $\inf A \in A$, then we call it the minimum of $A$ denoted by $\min A$.

\textbf{Ex.:} Let $\mathbb{R}\supseteq A=\{{1\over n} : n\in \mathbb{N}\}$. Then $\sup A=1$ belongs to $A$, while $\inf A=§$ does not belong to $A$.

\textbf{Def.:} let us introduce the elements $\infty$,$-\infty$, so that $\infty >x>-\infty$ for any $x\in \mathbb{R}$ and define the extended real numbers as $\overline{\mathbb{R}}=\{\infty, -\infty\}\cup \mathbb{R}$. If a set $A\subseteq \mathbb{R}$ is not bounded from above then define $\sup A=\infty$, and if $A\subseteq \mathbb{R}$ is not bounded from below then define $\inf A=-\infty$. Also define $\sup \emptyset=-\infty$ and $\inf \emptyset = \infty$.

\subsection{(Order) Completeness}

Consider $A=\{x\in \mathbb{Q}:x^2\leq 2\}$. This set is bounded from above but has no least upper bound in $\mathbb{Q}$.

\textbf{Def(Completeness).:} a totally ordered field $Z$ is complete, if all nonempty upper bounded subsets of $Z$ have a least upper bound in $Z$. We call this the least upper bound property.

\textbf{Theorem(Dedekind):} There exists a unique (up to $(\cdot, +,\leq)$-preserving transformation) ordered complete field satisfying the order axioms I., II. that contains $\mathbb{Q}$ and it is the field $\mathbb{R}$.
Such a transformation $\phi:\mathbb{R}\mapsto \mathcal{M}$ satisfies $\phi(x+y)=\phi(x)+\phi(y), \phi(xy)=\phi(x)\phi(y), x\leq y \implies \phi(x)\leq \phi(y)$.


\subsection{Archimedean property}

\textbf{Theorem(Archimedean property):} If $x\in \mathbb{R}$, then there exists $n\in \mathbb{Z}$ such that $x<n$.

\textbf{Proof:} Suppose, for contradiction, that there exists $x\in \mathbb{R}$ s.t. $x>n$ for all $n\in \mathbb{Z}$. Then $x$ is an upper bound of $\mathbb{Z\subseteq\mathbb{R}}$, so $M=\sup\mathbb{Z}\in\mathbb{R}$ exists. Since $n\leq M$ for all $n\in \mathbb{Z}$, we have $n-1\leq M-1$ for all $n\in \mathbb{Z}$, which implies $n\leq M-1$ for all $n\in \mathbb{Z}$. But then $M-1$ is an upper bound of $\mathbb{Z}$ that is strictly less than $M=\sup \mathbb{Z}$, a contradiction to $M=\sup \mathbb{Z}$ being the least upper bound.

\textbf{Corollary:} For every $0<\epsilon\in \mathbb{R}$, there exists an $n\in \mathbb{N}$, s.t. $0<{1\over n}<\epsilon$.

\textbf{Corollary(integer part):} If $x\in \mathbb{R}$, then there exists $[x]=n\in \mathbb{Z}$ called the integer part of $x$, such that $n\leq x<n+1$.


\subsection{Further properties}

\textbf{Def(dense set).:} $A\subseteq \mathbb{R}$ is dense in $\mathbb{R}$, if for any $0<\epsilon, x\in \mathbb{R}$ there exists $a\in A$, s.t. $x-\epsilon<a<x+\epsilon$.

\textbf{Theorem(density of rationals):} $\mathbb{Q\subseteq\mathbb{R}}$ is dense in $\mathbb{R}$.

\textbf{Proof:} Let $0<\epsilon, x\in \mathbb{R}$. Then for any $n\in N$ we have 
\[
[nx]\leq nx<[nx] + 1
\]
which gives
\[
{[nx]\over n}\leq x <{[nx]\over n} + {1\over n}
\]
Pick $n\in \mathbb{N}$, s.t. $0<{1\over n}<\epsilon$. Then we have
\[
{[nx]\over n}\leq x <{[nx]\over n} +{1\over n}<{[nx]\over n} +\epsilon
\]
which implies $x-\epsilon<{[nx]\over n} <x+\epsilon$ as wanted.

\subsection{Properties of sup and inf}



\subsection{Intervals and topology of $\mathbb{R}$}


\end{document}